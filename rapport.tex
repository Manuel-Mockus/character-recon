\documentclass[a4paper,11pt,twoside]{report}
% Ajouter l'option 'pdf' lorsque vous compilez avec pdflatex
\usepackage[french]{babel}
\usepackage[utf8]{inputenc}
\usepackage{graphicx} 
\usepackage{amsmath}
 

\author{...}
\title{Reconnaissance de caractères}

\begin{document}
 
\maketitle 

%\begin{enonce}
%...
%\end{enonce}

%\begin{resume}
%...
%\end{resume}

\chapter{Introduction}
Le coeur du projet est d'écrire un code permettant de reconnaître automatiquement des caractères manuscrits, le but étant de familiariser avec des méthodes usuelles de reconnaissance de forme en les implémentant pour un cas particulier et en en comprenant la théorie. \\

Les caractères à analyser seront ici des chiffres de 1 à 9 sous forme de vecteurs de taille $(28\times28,1)$ à valeurs entières comprise entre 0 et 255 : chaque vecteur représente une image de taille $28\times28$ en nuance de gris (0 = noir, blanc = 255). Nous disposons d'une base de données de 70000 chiffres manuscrits labelisés stockés sous le format décrit précédemment.\\
 
Nous avons choisi comme langage de programmation pour notre projet le Python.

\chapter{Pré-traitement des données}
La base de donnée étant un fichier matlab, nous avons dans un premier temps du la traduire dans un format utilisable sous python. Nous avons choisi comme format de stockage une liste de liste de vecteurs : le kième éléments de cette liste est une liste des vecteurs correspondants au chiffre k.\\

Les algorithmes étudiés relèvent du domaine de l'apprentissage automatique : après une phase d'apprentissage où l'on donne à l'algorithme des images et les chiffres respectifs comme références, on a une phase de traitement où l'algorithme doit reconnaître le chiffre à partir de l'image seule. Pour avoir des résultats pertinents, il faut donc distinguer clairement les données utilisées pour l'apprentissage et celles utilisées pour le traitement.\\(à compléter)\\

Nous avons ensuite divisé la base de donnée afin que chacun des membre du groupe puisse effectuer des tests indépendants. 


\chapter{Première méthode : calcul des écarts à la moyenne} 
\chapter{Deuxième méthode : utilisation de la décomposition en valeurs singulière (SVD)}
\chapter{Conclusion}

\bibliography{exemple} % Utilise exemple.bib
  
% Prints all the non-cited references
\nocite{*} 
% Use style 'alphakey' or 'alpha' for the draft, and then switch 
% to 'unsrt' or 'plain' or 'ieeetr' styles for the final version, 
% since they are the IEEE preferred ones. 
\bibliographystyle{ieeetr}
%\bibliographystyle{alpha}
\cleardoublepage

\appendix

\chapter{Annexe 1}
...
\end{document}
